\documentclass[a4paper]{article}
\usepackage[spanish,es-tabla]{babel}	% trabajar en español
\spanishsignitems	
%\usepackage{simplemargins}

%\usepackage[square]{natbib}
\usepackage{amsmath}
\usepackage{amsfonts}
\usepackage{amssymb}
\usepackage{bbold}
\usepackage{graphicx}
\usepackage{blindtext}
\usepackage{hyperref}
\usepackage{mathtools}
\usepackage{dirtytalk}

\begin{document}
\pagenumbering{arabic}

\Large
 \begin{center}
\textbf{Prueba del Teorema de Bell y violación desigualdad CHSH}\\
Seminario 3  

\hspace{10pt}

% Author names and affiliations
\large
Lic. Julio A. Medina$^1$ \\

\hspace{10pt}
\small  
$^1$ Universidad de San Carlos, Escuela de Ciencias Físicas y Matemáticas\\
Maestría en Física\\
\href{mailto:julioantonio.medina@gmail.com}{julioantonio.medina@gmail.com}\\

\end{center}

\hspace{10pt}


\normalsize

\begin{abstract}
El Teorema de Bell y las desigualdades asociadas fueron de gran importancia para establecer la validez de las correlaciones que se dan en la mecánica cuántica, con este se logró esclarecer la paradoja de Einstein-Podolsky-Rosen sobre teorías de variables ocultas y la no-localidad de la teoría cuántica. Para establecer la validez experimental de los resultados de Bell, Clauser, Horne, Shimony y Holt derivaron las desigualdades CHSH que al igual que las desigualdades de Bell poné restricciones en las ocurrencias estadísticas de una \say{prueba de Bell}. Estas confirmaciones experimentales pueden realizarse por medio de un circuito cuántico, en este reporte se expande en todo el desarrollo teórico y se implementan los circuitos por medio de Qiskit para comprobar que la naturaleza viola las desigualdades CHSH.
\end{abstract}

\section{Correlaciones en las mediciones de spin y las desigualdades de Bell}
El ejemplo más simple de adición de momento angular en sistemas compuestos de varias partículas en Mecánica Cuántica es el caso de spin $\frac{1}{2}$ ver por ejemplo \cite{Sakurai}, este sistema se usa para demostrar uno de los efectos de la mecánica cuántica más sorprendentes y que ha causado controversias y discusiones científicas famosas i.e. La paradoja de Eistein, Podolsky y Rosen, ver \cite{Einstein}, \cite{Bell}.\\

Considerando un sistema de dos electrones en un estado \textit{spin-singlet}, i.e. con spin total igual a $0$. El estado puede escribirse cómo 

\begin{equation}\label{eq::singlet_state}
|\text{\textit{spin-singlet}}\rangle=\frac{1}{\sqrt{2}}\big( |\mathbf{\hat{z}}+;\mathbf{\hat{z}}-\rangle -|\mathbf{\hat{z}}-;\mathbf{\hat{z}}+\rangle  \big)
\end{equation}
donde se ha especificado explícitamente la dirección de cuantización, para hacer una reseña se recuerda que en $|\mathbf{\hat{z}}+;\mathbf{\hat{z}}-\rangle$ se interpreta que el primer electrón está en el estado spin ''arriba'' y el segundo está en el estado spin ''abajo'' de manera análoga para $|\mathbf{\hat{z}}-;\mathbf{\hat{z}}+\rangle$ se tiene al primer electrón en el estado spin ''abajo'' y al el segundo está en el estado spin ''arriba''.\\
Ahora si se realiza una medición sobre el estado definido en \ref{eq::singlet_state} hay una probabilidad $p=0.5$ de encontrar al sistema en un estado ''arriba''  o ''abajo'' ya que el sistema tiene la misma probabilidad de estar en el estado $|\mathbf{\hat{z}}+;\mathbf{\hat{z}}-\rangle$ o $|\mathbf{\hat{z}}-;\mathbf{\hat{z}}+\rangle$. En esta medición si se encuentra a uno de los electrones con un spin ''arriba'' el otro necesariamente tiene que estar en con el spin ''abajo'' y viceversa. Cuando se halla al primer electrón en el estado spin ''arriba'' el aparato de medición a colapsado la función de onda del sistema al estado(en otras palabras el aparato de medición ha seleccionado) el primer término de \ref{eq::singlet_state}, $|\mathbf{\hat{z}}+;\mathbf{\hat{z}}-\rangle$ una medición subsecuente en el spin del segundo electrón debe reafirmar que estado compuesto está dado por $|\mathbf{\hat{z}}+;\mathbf{\hat{z}}-\rangle$.\\

Es impresionante que este tipo de correlación pueda persistir incluso cuando las 2 partículas del sistema estén bastante alejadas una de otro y hayan dejado de interactuar localmente dado que cuando mientras se alejan el movimiento no cambia el spin de las partículas. Este es el caso de un sistema con momento angular $J=0$ que se desintegra espontáneamente es dos partículas de spin $\frac{1}{2}$ sin momento angular orbital relativo, debido a que el momento angular se debe conservar en el proceso de desintegración. Un ejemplo experimental de esto se da en un proceso escaso o raro en el que un mesón $\eta$ con masa $549\frac{\text{MeV}}{c^2}$ decae un par de muones
\begin{equation}
\eta\rightarrow\mu^+ + \mu^-
\end{equation}

\begin{thebibliography}{99}
%% La bibliografía se ordena en orden alfabético respecto al apellido del 
%% autor o autor principal
%% cada entrada tiene su formatado dependiendo si es libro, artículo,
%% tesis, contenido en la web, etc
\bibitem{Arfken} George Arfken. \textit{Mathematical Methods for Physicists}.

\bibitem{Bell} J.S. Bell. \textit{On the Einstein Podolski Rosen Paradox}. \url{https://cds.cern.ch/record/111654/files/vol1p195-200_001.pdf}

\bibitem{Clauser} John F. Clauser, Michael A. Horne, Abner Shimony, Richard Holt. \textit{PROPOSED EXPERIMENT TO TEST LOCAL HIDDEN-VARIABLE THEORIES.}. Physical Review Letters,. 23(15):880-4, \url{https://journals.aps.org/prl/abstract/10.1103/PhysRevLett.23.880}.

\bibitem{Einstein} Einstein A., B. Podolsky, N. Rosen, \textit{Can Quantum-Mechanical Description of Physical Reality be Considered Complete?}. Physical Review. \url{doi:10.1103/PhysRev.47.777}

\bibitem{Medina} J. Medina. \textit{Reporte de Seminario 1. Computación Cuántica}. \url{https://github.com/Julio-Medina/Seminario/blob/main/Reporte_final/reporte_final.pdf}

\bibitem{Nielsen} Michael A. Nielsen, Isaac L. Chuang. \textit{Quantum Computation adn Quantum Information}. Cambridge University Press 2010. 10th. Anniversary Edition.

\bibitem{Feynman} Richard P. Feynman. \textit{Simulating Physics with Computers.} \url{https://doi.org/10.1007/BF02650179}.

\bibitem{Qiskit} \textit{Qiskit Textbook}. \url{https://qiskit.org/textbook-beta}

\bibitem{Mermin} N. David Mermin \textit{Quantum Computer Science: An Introduction}. Cambridge University Press, 2007.

\bibitem{Sakurai} J.J. Sakurai \textit{Modern Quantum Mechanics}. The Benjamin/Cummings Publishing Company, 1985.

\bibitem{Dotsenko} Viktor Dotsenko. \textit{An Introduction to the Theory of Spin Glasses and Neural Networks}. World Scientific 1994.

\bibitem{Bahri} Yasaman Bahri, Jonathan Kadmon, Jeffrey Pennington, Sam S. Schoenholz, Jascha Sohl-Dickstein, Surya Ganguli. \textit{Statistical Mechanics of Deep Learning}. \url{https://www.annualreviews.org/doi/pdf/10.1146/annurev-conmatphys-031119-050745}

\bibitem{openQASM} OpenQASM. \url{https://github.com/openqasm/openqasm}.
 

\end{thebibliography}
\end{document}

